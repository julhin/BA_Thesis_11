%% LaTeX2e class for student theses
%% sections/abstract_en.tex
%% 
%% Karlsruhe Institute of Technology
%% Institute for Program Structures and Data Organization
%% Chair for Software Design and Quality (SDQ)
%%
%% Dr.-Ing. Erik Burger
%% burger@kit.edu
%%
%% Version 1.3.2, 2017-08-01

\Abstract
In the last few the software systems became more complex and interconnected. Therefore, security in general and privacy in particular becomes more important. So the idea for a architectural security analysis emerged. To conduct an architectural security analysis one of many approaches is a data-based privacy analysis. Such approaches cannot be evaluated formally. Instead case studies are used for the evaluation. Since case studies create qualitative data, the creation of usable case studies for an evaluation is not trivial. In this thesis, we develop a procedure. With this procedure usable case studies for a software system may be generated. The security have to be modeled via access rights. We further applied the procedure to a software system and created a case study. The created case study consists of two parts. First, the defined access rights that are stored in a matrix and the system model extended by data flows. We then evaluated the created case study based on two aspect. First the quality of the defined access rights and, secondly, the covered information flow classes. Further, the procedure itself is evaluated regarding the applicability. Finally, the chosen modeling language is evaluated in terms of its expressiveness.