%% LaTeX2e class for student theses
%% sections/content.tex
%% 
%% Karlsruhe Institute of Technology
%% Institute for Program Structures and Data Organization
%% Chair for Software Design and Quality (SDQ)
%%
%% Dr.-Ing. Erik Burger
%% burger@kit.edu
%%
%% Version 1.3.2, 2017-08-01

\chapter{Introduction}
\label{ch:Introduction}
\section{Motivation}
In the course of the last few years, software systems became more interconnected and complex. Security in general and data protection in particular became increasingly important. Therefore, for software systems it became mandatory to ensure security. If a system does not ensure privacy a lot of damage could be done.  First of all, the financial loss through possible leaks are immense \cite{privHarm}. But the betrayal of the customer by not protecting their data might be the greater issue. Often this results in a loss of trust in the system. So, for all commercially deployed systems the guarantee of privacy should be a primary design goal. Privacy, on the other hand, is a non-functional requirement and it is difficult to ensure compliance. Due to the importance of privacy, the idea emerged to review security on an architectural level. These approaches offers two main benefits. Currently, the system model and the security documentation are stored at different places which leads to inconsistency between those two. IThere is no guarantee that changes to one model will be transferred to the other. Therefore, architectural security analysis store the two aspects, system model and security documentation, in one model. Further the system model can be adapted in an early stage. Early stage here refers to the stage before the implementation or the model is added to an already existing implementation. To conduct a architectural security analysis various approaches are available. One popular approach is UMLsec \cite{UMLSec}. This approach utilize the UML modeling language and extends the mate model to add security specifications. Then an attacker model is used to evaluate the security. Another approach is data-based approaches. Seifermann's approach \citep{Seifermann16}. The approach utilizes component-based systems by extending them with data flows. The security is evaluated by analyzing the data flows with a constraint solving system.\\ One big disadvantage of the data-based security analysis approaches is that these approaches cannot be validated by a formal proof. Therefore, case studies are used. It is not trivial to create usable case studies for evaluation. Case studies are qualitative data and therefore require a lot of effort in categorizing and analyzing all aspects of a specific case. The goal of this thesis is to support the process for creating case studies. We introduced a procedure to create case studies for a component-based system with the restriction that the security is modeled based on access rights. The thesis was evaluated based on three aspects:
\begin{itemize}
\item the introduced procedure.
\item the used modeling language for component-based systems.
\item the created example case study by applying the procedure to a system. The case study has been evaluated based on two aspects: the defined access rights and the covered information flow classes.
\end{itemize}
%Ergebnisse
We verified the applicability of the introduced procedure by applying it to the Component Community Modeling Example (CoCoME). We chose data-centric Palladio Component Model (PCM) as our modeling language. We also verified that PCM is able to express the case study. Nonetheless, currently it is rather cumbersome to add access rights to the model. Therefore, a meta model element for the access rights is missing. At last , we evaluated the created case study. The defined access rights achieved to cover about 43\% of the access right criteria introduced by Evered and Bögeholz \cite{CaseStudyAndAccessrigths}. Further, half of the defined information flow classes for Non-influence \citep{Noninfluence} are covered int he case study.

\section{Contributions}
The two main contribution of the thesis are the presented procedure and the created case study by applying the procedure to CoCoME.\\
We presented a detailed procedure for creating a case study for component-based systems. The general process is derived from Runeson and Höst \cite{CaseStudysoftwaresystems}. We refined general process so that it is usable to create 


First of all, we defined a method to create case studies that may be used to conduct and validate data-based privacy analyses on an architectural level. Also we defined an evaluation for the created case study. The evaluation allows to validate the quality of defined access rights and the covered information flow classes. We created a sample case study from CoCoME\cite{CoCoMETechReport} and evaluated it. In procedure of creating the case study from CoCoME, we discovered some vagueness in CoCoME. Some data types were not defined at all, for example. To fix this, we provided definitions for all vague elements of CoCoME that we used in our case study. The definitions mainly include roles, data and access rights. Further, we used a meta model extension for the Palladio Component Model (PCM)\cite{PCM} provided by Seifermann\cite{MMextension} to model data flows in the resulting case study. We have evaluated the state of PCM with the metamodel extension to see if it is possible to use it in the case study.
\section{Outline of the thesis}
The thesis is organized as follows. In \autoref{ch:basiscs}, we introduce the foundations. In \autoref{ch:method}, the method we are using to create a case study is presented. Next, we apply the introduced method over the course of two chapters. First, we analyze CoCoME (\autoref{ch:cocome}) and define the missing elements and add it to the system, secondly, we create the case study in \autoref{casestudysystem}. In \nameref{ch:eval} we evaluate the method, case study and the used modeling language and discuss the evaluation. In \autoref{ch:relWork}, we put the contributions in the context of related work. In the \autoref{ch:Conclusion}, we conclude the thesis and discuss the future work.