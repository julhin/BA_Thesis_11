%% LaTeX2e class for student theses
%% sections/content.tex
%% 
%% Karlsruhe Institute of Technology
%% Institute for Program Structures and Data Organization
%% Chair for Software Design and Quality (SDQ)
%%
%% Dr.-Ing. Erik Burger
%% burger@kit.edu
%%
%% Version 1.3.2, 2017-08-01

\chapter{Introduction}
\label{ch:Introduction}
\section{Motivation}
In the course of the last few years, software systems became more interconnected and complex. Security in general and data protection in particular became increasingly important. Therefore, for software systems it became mandatory to ensure security. If a system does not ensure privacy a lot of damage could be done.  First of all, the financial loss through possible leaks are immense \cite{privHarm}. But the betrayal of the customer by not protecting their data might be the greater issue. Often this results in a loss of trust in the system. So, for all commercially deployed systems the guarantee of privacy should be a primary design goal. Privacy, on the other hand, is a non-functional requirement and it is difficult to ensure compliance. Due to the importance of privacy, the idea emerged to review security on an architectural level. These approaches offers two main benefits. Currently, the system model and the security documentation are stored at different places which leads to inconsistency between those two. IThere is no guarantee that changes to one model will be transferred to the other. Therefore, architectural security analysis store the two aspects, system model and security documentation, in one model. Further the system model can be adapted in an early stage. Early stage here refers to the stage before the implementation or the model is added to an already existing implementation. To conduct a architectural security analysis various approaches are available. One popular approach is UMLsec \cite{UMLSec}. This approach utilize the UML modeling language and extends the mate model to add security specifications. Then an attacker model is used to evaluate the security. Another approach is data-based approaches. Seifermann's approach \cite{Seifermann16}. The approach utilizes component-based systems by extending them with data flows. The security is evaluated by analyzing the data flows with a constraint solving system.\\ One big disadvantage of the data-based security analysis approaches is that these approaches cannot be validated by a formal proof. Therefore, case studies are used. It is not trivial to create usable case studies for evaluation. Case studies are qualitative data and therefore require a lot of effort in categorizing and analyzing all aspects of a specific case. The goal of this thesis is to support the process for creating case studies. We introduced a procedure to create case studies for a component-based system with the restriction that the security is modeled based on access rights. The thesis was evaluated based on three aspects:
\begin{itemize}
\item the introduced procedure.
\item the used modeling language for component-based systems.
\item the created example case study by applying the procedure to a system. The case study has been evaluated based on two aspects: the defined access rights and the covered information flow classes.
\end{itemize}
%Ergebnisse
We verified the applicability of the introduced procedure by applying it to the Component Community Modeling Example (CoCoME). We chose data-centric Palladio Component Model (PCM) as our modeling language. We also verified that PCM is able to express the case study. Nonetheless, currently it is rather cumbersome to add access rights to the model. Therefore, a meta model element for the access rights is missing. At last , we evaluated the created case study. The defined access rights achieved to cover about 43\% of the access right criteria introduced by Evered and Bögeholz \cite{CaseStudyAndAccessrigths}. Further, half of the defined information flow classes for Non-influence \cite{Noninfluence} are covered int he case study.

\section{Contributions}
The two main contribution of the thesis are the presented procedure and the created case study by applying the procedure to CoCoME.\\
We presented a detailed procedure for creating a case study for component-based systems. The general process is derived from Runeson and Höst \cite{CaseStudySoftware}. We have refined the general process so that it is possible to create case studies of component-based systems. Further, inside the procedure, we defined evaluation points during the procedure for verifying the quality of the created case study. \\ 
Second, we created a case study for CoCoME. During the process, we defined system extensions for CoCoME to clearly define some vague system elements. We have also created a concrete example for access rights in component-based systems according to the definition of Evered and Bögeholz \cite{CaseStudyAndAccessrigths}. At last, we identified and defined data flows for CoCoME and added the data flows to the system model.
\section{Outline of the thesis}
The thesis is organized as follows. In \autoref{ch:basiscs} the foundations for the thesis are presented. This includes the used terminology, the foundations for case study creation and the used evaluation method. In \autoref{relWork} we put the thesis in the context of related work. The in \autoref{ch:method} the procedure is presented in detail. \autoref{ch:cocome} then analyzes the current status of CoCoME and defines corresponding system extensions. The chapter concludes with an evaluation of the access rights. Afterwards, in \autoref{ch:casestudysystem}, the case study is created. The chapter concludes with an evaluation of the defined data flows. Then, in \autoref{ch:eval} the three aspects of the thesis are evaluated. Further, the threats to validity are discussed. At last, the thesis is concluded in \autoref{ch:Conclusion} with a discussion of the results and the possible future work.