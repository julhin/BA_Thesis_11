%% LaTeX2e class for student theses
%% sections/conclusion.tex
%% 
%% Karlsruhe Institute of Technology
%% Institute for Program Structures and Data Organization
%% Chair for Software Design and Quality (SDQ)
%%
%% Dr.-Ing. Erik Burger
%% burger@kit.edu
%%
%% Version 1.3.2, 2017-08-01

\chapter{Conclusion and future work}
\label{ch:Conclusion}
In this chapter, we first discuss in which cases our work can be used and then the future work for each aspect of the evaluation. Lastly a conclusion is drawn.
\section{Future work} 
\label{FW}
This section presents possible approaches for future work. The discussion of future work will be divided into the three main aspects of the thesis: the method for creating a case study, the resulting case study and its evaluation, and at last the used modeling language PCM.. 
\subsection{Procedure}
In the case of the procedure, we validated the applicability for an excerpt of CoCoME. The next step is to create a case study for the entire CoCoME system. Further, the approach should be applied to other systems covering areas other than CoCoME. A concrete example would be a travel system \cite{Travelsystem}.
\subsection{Case study system}
For the case study, we decided to divide the future work in a short term and a long term work.\\
\paragraph{Short term work}
As short term work, one should define more scenarios. First, to cover all defined information flow classes. Also another possibility is to define scenarios that for each information flow class a violation and a counterpart is defined. Furthermore, a survey should be conducted to evaluate the access rights clear and concise. After the survey is done, the case study should be reviewed.
\paragraph{Long term work}
A long term work would be the addition of the access rights into the CoCoME implementation to allow an evaluation of the criteria \textit{fundamental} and \textit{efficient}. Further, additional information flow classes, that are not yet covered by \textit{Non-influence} and define corresponding scenarios. At last, an evaluation for a data-based privacy approach should be conducted by using the case study. 
\subsection{Meta model extension}
As future work, we propose to extend the meta model of PCM even further to allow the storage of the ACM in the same model to negate possible inconsistencies.
\section{Conclusion}
%TODO hier weiter mchen !!!!
In this thesis, we presented a method to create case studies for a data-based privacy analysis. To allow such analyses, we defined data flows for a system. Further, we created a process to create case studies from software systems and applied the method to CoCoME. While in the process of application to CoCoME, we found vagueness in CoCoME. This vagueness does not happen by accident. Another benefit of the method, beside the creation of a case study, is to detect vague definitions and add definitions for them . First of all, the roles were rather well defined, but missed the last bit of precision. For example, the role of StockManager was mentioned, but there were too few use cases to derive tasks and provided/required data. Also for some data used in CoCoME exact definitions were missing. After we added the definitions, data flows that describe the data processing were created. Further, we define a new use case (UC14) for CoCoME. The UC14 introduces a new role, the support employee. When a user encounter problems with an order they can issue a ticket. The support employee handles the tickets to solve the problem.Then we evaluated the created case study based on two aspects. First, the quality of the defined access rights, secondly, the number of different information flow classes covered. The created case study  covers 50\% of the investigated information flow classes. We then validated the method described through the application to CoCoME. We have found that the method is applicable to CoCoME, but still needs to be applied to other component-based systems.  \autoref{FW} goes into more detail on which types of systems the method should be applied.  At last we evaluated the used modeling language PCM \cite{PCM}. We used a meta model extension to add the data flows to the system model. The meta model extension allows the modeling of data flows, but misses an element to store the ACM in the same model.\\
At last, CoCoME was not the perfect system for this thesis. First of all, CoCoME is in constant development and therefore is more a proof of concept than a full system. Further, before the addition of the PickupShop, CoCoME was a closed system with a finite number of users. With the addition, CoCoME became an open system, therefore we chose the PickupShop as our system to perform the procedure on.
\section{Benefits of our work}
The main goal of the thesis was to introduce a method to create case studies that may be used to evaluate approaches for privacy analyses on an architectural level. More specific, we aimed to create case studies that can be used by data-based privacy analyses for validation. The created sample case study is a ready-to-go case study on which an data-based privacy analyses approach may be tested. Ready-to-go means that the case study may not be perfect but may be used to evaluate data-based privacy analysis approaches in the current state.  Further, we provided evaluation criteria to measure the quality of the created case study. The main benefit of this work is to have a process for transforming software systems to perform a data-based data privacy analysis. The process benefits motivated software architects, who want to perform a data-based privacy analysis in an early stage of the design process to ensure compliance for the system.

