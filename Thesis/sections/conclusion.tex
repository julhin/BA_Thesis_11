%% LaTeX2e class for student theses
%% sections/conclusion.tex
%% 
%% Karlsruhe Institute of Technology
%% Institute for Program Structures and Data Organization
%% Chair for Software Design and Quality (SDQ)
%%
%% Dr.-Ing. Erik Burger
%% burger@kit.edu
%%
%% Version 1.3.2, 2017-08-01

\chapter{Related work}
\label{ch:relWork}
After the contributions of the thesis have been evaluated, they are placed in the context of related works. First, it has to be said that related work for the kind of case study we created was hard to find. Case studies are an acknowledged procedure mainly in the fields of sociology, philosophy, law, etc. For this fields, the case study method is documented well. The type of case study that we are going to create shows some similarities with case studies that were created in the previously mentioned fields. Nevertheless, the discrepancy between computer science and the classical fields for case studies, is too great. Only fundamental methods could be adopted by the method. Such methods includes, for example, to use two sources of data \cite{cs2sources}. In the field of computer science, case studies are mostly used to give an overview of certain topics, e.g. how different challenges in software engineering were tackled. Sometimes there are used to verify the applicability of different approaches, like UMLSec \cite{UMLSecCS}. The only publication that roughly outlines the creation a case study for a software system, was the one from Evered and Bögeholz \cite{CaseStudyAndAccessrigths}. In this publication, the both authors describe briefly the creation of a case study on a much smaller system than we performed our case study on. The main goal of their publication was to define good access rights in component based systems by using their structure. We have adopted the introduced criteria of Evered and Bögeholz \cite{CaseStudyAndAccessrigths} and have included them in our evaluation as a measurement for good access rights. Further, the author described briefly the creation of a case study. In their publication, the defined first the data than the access rights and then went right to evaluation. In their work this was enough because the main aim of their work was to evaluate the different types how access rights may be added to a running system. In contrast to Evered and Bögeholz, the aim for our case study was to be used in a data-based privacy analysis. Therefore we used their approach as a basis for ours. We require a specific system model and the existence or the deductibility of use cases to ease up the later validation. Further, in contrast to Evered and Bögeholz, we want to use the created case study in another setting. Therefore we added the definition of scenarios and the definition of the type of data processing for each component and each role. All in all, we say that Evered and Bögeholz have given us a good starting point for our method and we have expanded theirs.\\ We lloked further but could not find any other fitting publications that may address our matter. We found a lot publications that adresses different case studies but none with the same or similar aim as our method.
\chapter{Conclusion and future work}
\label{ch:Conclusion}
In this chapter, we first draw a conclusion for the thesis. Then, we discuss in which cases our work can be used and lastly, we discuss the future work for each aspect of the evaluation.
\section{Conclusion}
In this thesis, we presented a method to create case studies for a data-based privacy analysis. To allow such analyses, we defined data flows for a system. Further, we created a process to create case studies from software systems and applied the method to CoCoME. While in the process of application to CoCoME, we found vagueness in CoCoME. This vagueness does not happen by accident. Another benefit of the method, beside the creation of a case study, is to detect vague definitions and add definitions for them . First of all, the roles were rather well defined, but missed the last bit of precision. For example, the role of StockManager was mentioned, but there were too few use cases to derive tasks and provided/required data. Also for some data used in CoCoME exact definitions were missing. After we added the definitions, data flows that describe the data processing were created. Further, we define a new use case (UC14) for CoCoME. The UC14 introduces a new role, the support employee. When a user encounter problems with an order they can issue a ticket. The support employee handles the tickets to solve the problem.Then we evaluated the created case study based on two aspects. First, the quality of the defined access rights, secondly, the number of different information flow classes covered. The created case study  covers 50\% of the investigated information flow classes. We then validated the method described through the application to CoCoME. We have found that the method is applicable to CoCoME, but still needs to be applied to other component-based systems.  \autoref{FW} goes into more detail on which types of systems the method should be applied.  At last we evaluated the used modeling language PCM \cite{PCM}. We used a meta model extension to add the data flows to the system model. The meta model extension allows the modeling of data flows, but misses an element to store the ACM in the same model.\\
At last, CoCoME was not the perfect system for this thesis. First of all, CoCoME is in constant development and therefore is more a proof of concept than a full system. Further, before the addition of the PickupShop, CoCoME was a closed system with a finite number of users. With the addition, CoCoME became an open system, therefore we chose the PickupShop as our system to perform the procedure on.
\section{Benefits of our work}
The main goal of the thesis was to introduce a method to create case studies that may be used to evaluate approaches for privacy analyses on an architectural level. More specific, we aimed to create case studies that can be used by data-based privacy analyses for validation. The created sample case study is a ready-to-go case study on which an data-based privacy analyses approach may be tested. Ready-to-go means that the case study may not be perfect but may be used to evaluate data-based privacy analysis approaches in the current state.  Further, we provided evaluation criteria to measure the quality of the created case study. The main benefit of this work is to have a process for transforming software systems to perform a data-based data privacy analysis. The process benefits motivated software architects, who want to perform a data-based privacy analysis in an early stage of the design process to ensure compliance for the system.
\section{Future work} 
\label{FW}
This section presents possible approaches for future work. The discussion of future work will be divided into the three main aspects of the thesis: the method for creating a case study, the resulting case study and its evaluation, and at last the used modeling language PCM.. 
\subsection{Method}
In the case of the method, we validated the applicability for CoCoME. The next logical step is to apply the described method to other systems to verify the applicability. CoCoME  sells products, either via the PickupShop or a cash desk. Possible system to apply the method are systems which also handles a warehouse and/or ships the product to the customer. It would also be conceivable to apply the method to systems away from the supermarket scenario, for example, a flight booking system.
\subsection{Case study system}
For the case study, we decided to divide the future work in a short term and a long term work.\\
As short term work, one should define more scenarios. First, to cover all defined information flow classes. Also for each information flow class, one should define more scenarios to allow a more deep analysis. Another possible approach for the future work is the analysis of access rights according to various criteria not yet covered. We could not cover, for example,fundamental and efficient. As future work, we propose to include access rights in the current implementation. This makes it possible to check whether the access rights have been added directly in the code and whether the checks of the access rights do not cause too much overhead.  The remaining access rights were not checked due to time constraints and the lack of running code with built-in access rights.\\ 
As a long-term work, we propose to perform analysis for the created case study in this thesis. One should define more not yet covered information classes . These additional information classes are added to allow a wider spread more in depth evaluation. Another possibility is to analyze the case study created in this thesis by using it to validate a data-based approach privacy analysis approach. The cases for additions or developments are not dealt with in this thesis.
\subsection{Meta model extension}
As future work, we propose to extend the meta model of PCM even further to allow the storage of the ACM in the same model.
