%% LaTeX2e class for student theses
%% sections/conclusion.tex
%% 
%% Karlsruhe Institute of Technology
%% Institute for Program Structures and Data Organization
%% Chair for Software Design and Quality (SDQ)
%%
%% Dr.-Ing. Erik Burger
%% burger@kit.edu
%%
%% Version 1.3.2, 2017-08-01

\chapter{Conclusion and future work}
\label{ch:Conclusion}
As a conclusion, we shortly recapitulate each aspects presented in the previous chapters. At last, the future work is discussed.
\section{Conclusion}
The two main contributions of this thesis are a procedure for creating case studies and a case study created by using this procedure. First, The procedure allows to create case studies that later are used for evaluating data-based privacy analysis approaches. The resulting case study consists of two parts: first, the defined access rights and second, a system model extended with data flows. Then each part of the case study is evaluated. We used evaluation criteria to measure good access rights provided by Evered and Bögeholz \cite{CaseStudyAndAccessrigths} and the created data flows are evaluated based on the coverage of predefined information flow classes. Here we utilized the information flow classes that are covered by the problem statement \textit{Non-influence} \cite{NonInfluence}. Secondly, an application to CoCoME was conducted. During the application process, we found vagueness in CoCoME. Another benefit of the procedure, beside the creation of a case study, is to detect vague definitions in the system and provide definitions for them . In CoCoME, the roles were rather well defined, but missed the last bit of precision. For instance, the role of the \textit{StockManager} was mentioned, but there were too few system elements to derive exact tasks and the provided/required data. Also for some data used in CoCoME exact definitions were missing. After we added the definitions, scenarios were defined. From this scenarios the data flows that describe the data processing in CoCoME, were derived. Further, we define a new use case (UC14) for CoCoME. The UC14 introduces a new role, the support employee. When a user encounter problems with an order, they can issue a ticket. The support employee handles the tickets and solve the problem. At the end we evaluated the case study and discussed the findings. At last, we discussed if the procedure for creating a case study may be concluded. Finally, the applicability of the procedure and the expressiveness of the chosen modeling language were evaluated.\\ We discovered that the procedure is applicable for CoCoME and the meta model extension is able to express the data flows, but it is not possible to store an ACM in the same model. For the case study, we achieved to cover about 43\% of the discussed access rights criteria and we covered 50\% of the presented information flow classes. Note that, case studies are qualitative data and therefore the numbers does not matter that much. In the case of the access rights each missing category in the evaluation have an impact on the case study. The desired coverage of different information flow classes is dependent on the objective that is planned to achieve with the case study. Therefore, no generalized for the coverage of the information flow classes may be made. \\
At last, CoCoME was not the most fitting system for this thesis, but the only one available that is sufficiently documented. First of all, CoCoME is in constant development and therefore is more a proof of concept than a full system. 
In addition, CoCoME was a closed system before the \textit{PickupShop} was introduced. There was only a limited number of users. Therefore the \textit{PickupShop} is also the part of CoCoME that is relevant for a privacy analysis. Therefore we are only working on an excerpt from CoCoME.
\section{Benefits of our work}
%TODO benefits in 2 teile -> procedure and application 
As stated beforehand, the two contributions of this thesis are the procedure for creating case studies and an example case study created with the procedure. 
\paragraph{Procedure} 
The procedure was derived from the more general process described by Runeson and Höst \cite{CaseStudySoftware}. We introduced a procedure that allows to create case studies usable for evaluating a data-based privacy analysis. In the procedure requirements for a system are defined. Further, for each of the two parts the evaluation is defined. The resulting case studies improve the process for evaluating data-based privacy analysis. With case studies created with a defined process the different evaluation of approaches may be evaluated and then compared.

\paragraph{Case study}
We applied the procedure and created a case study base don CoCoME. In this case study, we have shown how an ACM can be structured for component-based systems. The idea of the matrix was introduced by Evered and Bögeholz \cite{CaseStudyAndAccessrigths}. Also, CoCoME was extended to meet the necessary requirements for the creation of a case study. Therefore, we categorized the data, added definitions for the different roles and even added a new use case to CoCoME. Further, the system model were extended by the defined data flows. An important property of the defined data flows is that the data flows cover a violation of two information flow classes. A counterpart for the violation has also been defined. In an evaluation, this makes it possible to check a data-based approach. An approach should correctly categorize the violation and the counterpart. Also, the new definitions for CoCoME eliminate some vagueness in the system.
\section{Future work} 
\label{FW}
This section presents possible approaches for future work. The discussion of future work will be divided into the three main aspects of the thesis: the method for creating a case study, the resulting case study and its evaluation, and at last the used modeling language PCM.. 
\subsection{Procedure}
In the case of the procedure, we validated the applicability for an excerpt of CoCoME. The next step is to create a case study for the entire CoCoME system. Further, the approach should be applied to other systems covering areas other than CoCoME. A concrete example would be a travel system \cite{Travelsystem}.
\subsection{Case study system}
For the case study, we decided to divide the future work in a short term and a long term work.\\
\paragraph{Short term work}
As short term work, one should define more scenarios. First, to cover all defined information flow classes. Also another possibility is to define scenarios that for each information flow class a violation and a counterpart is defined. Furthermore, a survey should be conducted to evaluate the access rights clear and concise. After the survey is done, the case study should be reviewed.
\paragraph{Long term work}
A long term work would be the addition of the access rights into the CoCoME implementation to allow an evaluation of the criteria \textit{fundamental} and \textit{efficient}. Further, additional information flow classes, that are not yet covered by \textit{Non-influence} and define corresponding scenarios. At last, an evaluation for a data-based privacy approach should be conducted by using the case study. 
\subsection{Meta model extension}
As future work, we propose to extend the meta model of PCM even further to allow the storage of the ACM in the same model to negate possible inconsistencies.
