%% LaTeX2e class for student theses
%% sections/abstract_de.tex
%% 
%% Karlsruhe Institute of Technology
%% Institute for Program Structures and Data Organization
%% Chair for Software Design and Quality (SDQ)
%%
%% Dr.-Ing. Erik Burger
%% burger@kit.edu
%%
%% Version 1.3.2, 2017-08-01

\Abstract
Die Arbeit stellt eine Methode vor, die in der datenbasierte Datenschutzanalyse verwendet werden kann. Dazu wurde ein bekanntes System benutzt, auf das die Mehtode angewendet wurde. Als Ergebnis der Anwendung der Methode wurde eine Fallstudie erstellt. Zum Schluss haben wir die Methode, die Fallstudie sowie die verwendere Modellierungssprachen evaluiert. Die Ergebnisse der Evaluation sind die Folgenden: die Anwendbarkeit für ein spezielles System ist gegeben. Die erstellte Fallstudie deckt nicht alle überprüften Informationsflussklassen ab. Die verwendete Modellierungssprache ist bis auf eine kleine Einschränkung nutzbar. Zu guter letzt, die Ergbenisse der Arbeit können für datenbasierte Datenschutzanalysen zur Evaluation verwendet werden, um eine oder mehrere  Fallstudien  zu erstellen und deren Qualität zu überprüfen.