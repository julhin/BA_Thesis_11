%% LaTeX2e class for student theses
%% sections/abstract_de.tex
%% 
%% Karlsruhe Institute of Technology
%% Institute for Program Structures and Data Organization
%% Chair for Software Design and Quality (SDQ)
%%
%% Dr.-Ing. Erik Burger
%% burger@kit.edu
%%
%% Version 1.3.2, 2017-08-01

\Abstract
In den letzten Jahren wurden die Softwaresysteme immer komplexer und vernetzter. Daher wird die Sicherheit im Allgemeinen und die Privatsphäre im Besonderen immer wichtiger. So entstand die Idee zu einer architekturbasierten Sicherheitsanalyse.  Um eine architekturbasierte Sicherheitsanalyse durchzuführen, ist einer von vielen Ansätzen eine datenbasierte Datenschutzanalyse. Solche Ansätze können nicht formal bewertet werden. Stattdessen werden für die Auswertung Fallstudien verwendet. Da Fallstudien qualitative Daten liefern, ist die Erstellung von brauchbaren Fallstudien für eine Bewertung nicht trivial. In dieser Arbeit entwickeln wir ein Verfahren. Mit diesem Verfahren können brauchbare Fallstudien für ein Softwaresystem erstellt werden. Die Sicherheit muss über Zugriffsrechte abgebildet werden. Wir haben das Verfahren auch auf ein Softwaresystem angewendet und eine Fallstudie erstellt. Die erstellte Fallstudie besteht aus zwei Teilen. Erstens, den definierten Zugriffsrechte, die in einer Matrix gespeichert sind und das  um Datenflüsse erweiterterte Systemmodell . Wir haben dann die erstellte Fallstudie anhand von zwei Aspekten bewertet. Zum einen die Qualität der definierten Zugriffsrechte und zum anderen die abgedeckten Informationsflussklassen. Weiterhin wird das Verfahren selbst hinsichtlich der Anwendbarkeit bewertet. Schließlich wird die gewählte Modellierungssprache hinsichtlich ihrer Aussagekraft bewertet.
